% arara: clean: { extensions: [ aux, bbl, blg, fdb_latexmk, fls, log, out, synctex.gz, toc ]}

\documentclass[12pt,a4paper]{article}


%== \Usepackages ==========================================
%-- Standard and geometry ---------------------------------
\usepackage[utf8]{inputenc}
\usepackage[T1]{fontenc}
\usepackage[english]{babel}
\usepackage[margin=2.5cm]{geometry}
\usepackage{enumitem}
\usepackage{graphicx}
\usepackage{caption}
\usepackage{subcaption}
\usepackage{booktabs}
\graphicspath{{../img/}}

%-- Links and drawing -------------------------------------
\usepackage[colorlinks]{hyperref}
\usepackage{tikz}

%-- Math --------------------------------------------------
\usepackage{amsmath,amssymb,amsfonts,amsthm}
\usepackage{mathtools,mathrsfs}
\usepackage[capitalize,nameinlink]{cleveref}
\numberwithin{equation}{section}
\allowdisplaybreaks{}
\raggedbottom{}

%-- Draft utilities ---------------------------------------
% \usepackage[right]{showlabels}
% \usepackage[mathlines]{lineno}
% \linenumbers%\modulolinenumbers[5]


%== Colors ================================================
%-- \definecolor ------------------------------------------
\definecolor{coGB}{HTML}{1F77B4}
\definecolor{coWG}{HTML}{2CA02C}
\definecolor{coAR}{HTML}{D62728}

\newcommand{\Rd}{\color{coAR}}
\newcommand{\Bu}{\color{coGB}}
\newcommand{\Gr}{\color{coWG}}
\newcommand{\Bk}{\color{black}}

%--\hypersetup -------------------------------------------
\hypersetup{
    colorlinks = True,
    linkcolor  = coAR,
    citecolor  = coWG,
    urlcolor   = coGB,
}


%== Math ==================================================
%-- \mathbb -----------------------------------------------
\newcommand{\bbC}{\mathbb{C}}
\newcommand{\bbN}{\mathbb{N}}
\newcommand{\bbR}{\mathbb{R}}
\newcommand{\bbS}{\mathbb{S}}
\newcommand{\bbZ}{\mathbb{Z}}

%-- \mathscr ----------------------------------------------
\newcommand{\scrC}{\mathscr{C}}

%-- \mathrm -----------------------------------------------
\newcommand{\rmL}{\mathrm{L}}
\newcommand{\rmH}{\mathrm{H}}

%-- Constants ---------------------------------------------
\newcommand{\ex}{\mathsf{e}}
\newcommand{\im}{\mathsf{i}\mkern1mu}
\newcommand{\jc}{\mathsf{j}\mkern1mu}

%-- Functions ---------------------------------------------
\DeclareMathOperator{\Ai}{\mathsf{Ai}}
\DeclareMathOperator{\Bi}{\mathsf{Bi}}

\DeclareMathOperator{\bJ}{\mathsf{J}}
\DeclareMathOperator{\bY}{\mathsf{Y}}
\newcommand{\hO}{\mathop{}\!\mathsf{H}^{(1)}}
\newcommand{\hT}{\mathop{}\!\mathsf{H}^{(2)}}

%-- Keywords ----------------------------------------------
\newcommand{\loc}{\mathrm{loc}}
\newcommand{\comp}{\mathrm{comp}}

%-- Operators ---------------------------------------------
\newcommand{\di}[1]{\mathop{}\!\mathrm{d}#1}
\DeclareMathOperator{\OO}{\mathcal{O}}
\DeclareMathOperator{\oo}{\mathcal{\scriptstyle{} O}}
\DeclareMathOperator{\Div}{div}

%-- Symbols -----------------------------------------------
\newcommand{\abs}[1]{\left\lvert#1\right\rvert}
\newcommand{\norm}[1]{\left\lVert#1\right\rVert}
\newcommand{\plr}[1]{\left(#1\right)}
\newcommand{\clr}[1]{\left[#1\right]}

\newcommand{\restr}[2]{\left. #1\right\rvert_{#2}}

\newcommand{\setst}[2]{\left\lbrace#1\ \middle\vert\ #2\right\rbrace}
\newcommand{\setwt}[2]{\left\lbrace#1\ {:}\ #2\right\rbrace}

%-- Vectors -----------------------------------------------
\newcommand{\vect}[1]{\boldsymbol{#1}}

%-- Specials ----------------------------------------------
\newcommand{\eps}{\varepsilon}


%== Environment ===========================================
\theoremstyle{definition}
\newtheorem{definition}{Definition}[section]
\newtheorem{notation}[definition]{Notation}
\newtheorem{assumption}[definition]{Assumption}

\theoremstyle{plain}
\newtheorem{lemma}[definition]{Lemma}
\newtheorem{theorem}[definition]{Theorem}
\newtheorem{corollary}[definition]{Corollary}

\theoremstyle{remark}
\newtheorem{remark}[definition]{Remark}
\newtheorem{example}[definition]{Example}


%== Title =================================================
\title{A Matlab/Octave toolbox for the computation of Bessel/Hankel
functions with large complex order}

\author{Zo{\"\i}s \textsc{Moitier}}

\date{\today}


%== Document ==============================================
\begin{document}

\maketitle

\begin{abstract}
    We code a Matlab/Octave toolbox to compute the Bessel and Hankel functions with large complex order.
\end{abstract}

% \tableofcontents

%============================
\section{How to use the toolbox}
%============================

It is governed by the CeCILL-C license under French law and abiding by the rules of distribution of free software.
It can be used, modified and/or redistributed under the terms of the CeCILL-C license as circulated by CEA, CNRS and INRIA at the following URL\@: \url{http://www.cecill.info}.

\bigskip

The goal of the toolbox is to compute the Bessel functions \( \bJ_\nu \)/\( \bY_\nu \) and the Hankel functions \( \hO_\nu \)/\( \hT_\nu \) when the order \( \nu \) is a large complex number in modulus.
The main functions of the toolbox are:
\begin{table}[!hbtp]
    \centering
    \renewcommand{\arraystretch}{1.2}
    \renewcommand{\tabcolsep}{1em}
    \begin{tabular}{ll}
        \toprule
        Function            & How to call it.
        \\\midrule
        \( \bJ_\nu(z) \)    & \verb|J  = besselj_cplx(v, z)|
        \\
        \( \bJ_\nu'(z) \)   & \verb|Jp = besseljp_cplx(v, z)|
        \\[1ex]
        \( \bY_\nu(z) \)    & \verb|Y  = bessely_cplx(v, z)|
        \\
        \( \bY_\nu'(z) \)   & \verb|Yp = besselyp_cplx(v, z)|
        \\[1ex]
        \( \hO_\nu(z) \)    & \verb|H1  = besselh_cplx(v, 1, z)|
        \\
        \( {\hO_\nu}'(z) \) & \verb|H1p = besselhp_cplx(v, 1, z)|
        \\[1ex]
        \( \hT_\nu(z) \)    & \verb|H2  = besselh_cplx(v, 2, z)|
        \\
        \( {\hT_\nu}'(z) \) & \verb|H2p = besselhp_cplx(v, 2, z)|
        \\\bottomrule
    \end{tabular}
    \caption{Link between Bessel/Hankel functions and their name in the code.}
\end{table}

The implementation is base on uniform asymptotic expansions for large order as describe in~\cite{Tem97}.
For now, we should expect good accuracy for \( \abs{\nu} \geq 20 \).

% %============================
% \section{Numerical validation}
% %============================

% For numerical validation, we use Mathematica as a reference.
% The first test we did was to compare the values for 2000 random \( \nu = r_\nu \ex^{\im \theta_\nu} \) with \( r_\nu \in [400,40000] \), \( \theta_\nu \in \left[-\frac{\pi}{10^4},\frac{\pi}{10^4}\right] \) and for each value of \( \nu \) we took a random \( z_\nu \in [0.999 |\nu|, 1.001 |\nu|] \).
% \begin{figure}[h]
%     \centering
%     \includegraphics[scale=0.5]{err2000nuz.png}
%     \caption{In blue we have the 2000 \( \nu \) that we take and in red it is the \( z \).}
% \end{figure}
% For the 2000 couples \( (\nu,z_\nu) \) we computed the result of the two Bessel and Hankel functions with Mathematica and with my script, and we compute the relative error between them.
% The results are shown \cref{err2000}.
% \begin{figure}[ht]
%     \centering
%     \begin{subfigure}[h]{0.45\textwidth}
%         \includegraphics[width=\textwidth]{err2000j.png}
%         \caption{Relative error of J}
%     \end{subfigure}
%     \begin{subfigure}[h]{0.45\textwidth}
%         \includegraphics[width=\textwidth]{err2000y.png}
%         \caption{Relative error of Y}
%     \end{subfigure}

%     \begin{subfigure}[h]{0.45\textwidth}
%         \includegraphics[width=\textwidth]{err2000h1.png}
%         \caption{Relative error of H1}
%     \end{subfigure}
%     \begin{subfigure}[h]{0.45\textwidth}
%         \includegraphics[width=\textwidth]{err2000h2.png}
%         \caption{Relative error of H2}
%     \end{subfigure}
%     \caption{I plot the projection of \( (\Re(\nu),\Im(\nu),\text{relative error}) \) on the plan \( \Im(\nu)=0 \).}
%     \label{err2000}
% \end{figure}
% The maximum of the relative error is -7.707 in logarithmic scale. So we have 7 digits accurate but if we look in a smaller zone we can have more digits correct.
% We did a comparison in three other zone.

% \bigskip

% In zone 1 we have 1600 points with \( \Re(\nu) \in [395,405] \), \( \Im(\nu) \in [-10^{-5},-10^{-6}] \) and \( z = 414.93 \).
% The result are shown in figure \ref{err2}.
% The maximum of relative error is -10.634 in logarithmic scale.
% \begin{figure}[ht]
%     \centering
%     \begin{subfigure}[h]{0.45\textwidth}
%         \includegraphics[width=\textwidth]{err2j.png}
%         \caption{Relative error of J in zone 1}
%     \end{subfigure}
%     \begin{subfigure}[h]{0.45\textwidth}
%         \includegraphics[width=\textwidth]{err2y.png}
%         \caption{Relative error of Y in zone 1}
%     \end{subfigure}

%     \begin{subfigure}[h]{0.45\textwidth}
%         \includegraphics[width=\textwidth]{err2h1.png}
%         \caption{Relative error of H1 in zone 1}
%     \end{subfigure}
%     \begin{subfigure}[h]{0.45\textwidth}
%         \includegraphics[width=\textwidth]{err2h2.png}
%         \caption{Relative error of H2 in zone 1}
%     \end{subfigure}
%     \caption{The relative error of the Bessel and Hankel functions.}
%     \label{err2}
% \end{figure}

% \bigskip

% In zone 2 we have 1600 points with \( \Re(\nu) \in [3100,3200] \), \( \Im(\nu) \in [-0.7,0.5] \) and \( z = 3160.2 \).
% The results are shown in figure \ref{err3}.
% The maximum of relative error is -9.727 in logarithmic scale.
% \begin{figure}[ht]
%     \centering
%     \begin{subfigure}[h]{0.45\textwidth}
%         \includegraphics[width=\textwidth]{err3j.png}
%         \caption{Relative error of J in zone 2}
%     \end{subfigure}
%     \begin{subfigure}[h]{0.45\textwidth}
%         \includegraphics[width=\textwidth]{err3y.png}
%         \caption{Relative error of Y in zone 2}
%     \end{subfigure}

%     \begin{subfigure}[h]{0.45\textwidth}
%         \includegraphics[width=\textwidth]{err3h1.png}
%         \caption{Relative error of H1 in zone 2}
%     \end{subfigure}
%     \begin{subfigure}[h]{0.45\textwidth}
%         \includegraphics[width=\textwidth]{err3h2.png}
%         \caption{Relative error of H2 in zone 2}
%     \end{subfigure}
%     \caption{The relative error of the Bessel and Hankel functions.}
%     \label{err3}
% \end{figure}

% \bigskip

% In zone 3 we have 1600 points with \( \Re(\nu) \in [412.93,416.93] \), \( \Im(\nu) \in [-10^{-5},-10^{-6}] \) and \( z = 414.93 \).
% The resulte are shown in figure \ref{err4}.
% The maximum of relative error is -8.814 in logarithmic scale.
% \begin{figure}[ht]
%     \centering
%     \begin{subfigure}[h]{0.45\textwidth}
%         \includegraphics[width=\textwidth]{err4j.png}
%         \caption{Relative error of J in zone 3}
%     \end{subfigure}
%     \begin{subfigure}[h]{0.45\textwidth}
%         \includegraphics[width=\textwidth]{err4y.png}
%         \caption{Relative error of Y in zone 3}
%     \end{subfigure}

%     \begin{subfigure}[h]{0.45\textwidth}
%         \includegraphics[width=\textwidth]{err4h1.png}
%         \caption{Relative error of H1 in zone 3}
%     \end{subfigure}
%     \begin{subfigure}[h]{0.45\textwidth}
%         \includegraphics[width=\textwidth]{err4h2.png}
%         \caption{Relative error of H2 in zone 3}
%     \end{subfigure}
%     \caption{The relative error of the Bessel and Hankel functions.}
%     \label{err4}
% \end{figure}

% \bigskip

% To conclude, if \( |\nu| \ge 400 \) we have 7 digits correct and we can have more the greater \( |\nu| \) is and \( z \) not too close to \( \nu \).

%== Bibliography ==========================================
\bibliographystyle{abbrvurl}
\bibliography{references}

\end{document}
